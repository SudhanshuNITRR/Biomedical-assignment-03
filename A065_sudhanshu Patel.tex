\documentclass{article}
\usepackage[pdftex]{graphicx}
% Comment the following line to NOT allow the usage of umlauts
\usepackage[utf8]{inputenc}
% Uncomment the following line to allow the usage of graphics (.png, .jpg)
%\usepackage{graphicx}

% Start the document
\begin{document}

% Create a new 1st level heading
\title{FUTURE OF HEALTHCARE} 

1. INTRODUCTION:\\ \\
        Looking at the changes that took place in the field of health from the 19th century to the 20th century, it seems that the future of healthcare will be more developed and all work will be done digitally.Many things are constantly being discovered in the healthcare system and the best equipment and medicines are constantly being discovered.  If this is how the healthcare system continues to develop, one day it will come that it will be possible to cure all the diseases of the world.  Patients suffering from many diseases will be disease free.The death toll from any disease will be reduced. In future, dangerous diseases like covid, Swine Flu and Omicron will be given a befitting reply.  The cure for all diseases is still being searched by our scientists.
\\ In future all work will start going digital.  Patients can be treated sitting at home.  Robots and digital machines will be used instead of doctors in hospitals.  This will reduce the work of doctors and nurses. \\
\\
2. MEDICAL EQUIPMENTS IN FUTURE:
\\

In the olden times, the medical equipments were of very large size, as the development in the medical field continued, the equipments were developed in better quality and smaller sizes than before.  In future the equipment will become so small that it can be carried anytime and anywhere, that is, all the devices are being made portable.\\


\includegraphics[scale=0.4] {IMG-20220205-WA0002.jpg}
\\

Examples of portable devices:\\
a) Blood pressure Monitor\\
b) Glucose Monitoring System\\
c) Automated insulin pumps\\
d) Portable EKG/ECG Monitors\\
e) Key consideration for portable medical device development etc.\\


\includegraphics[scale=0.3]{IMG-20220205-WA0000.jpg}
\\
\\

3. SMARTPHONE APPLICATIONS IN MEDICINAL FIELD:\\

Smartphone applications will also continue to grow in the future, with medical devices becoming more portable and smarter.  All the work can be done at home without any doctors and hospital patients. The introduction of mobile applications to the medical industry has radically changed users' views about how to monitor medical conditions, diet, and nutrition quickly and effectively. In addition to simple calorie calculators and heart monitors, mobile applications can now check medical records that doctors make in hospitals. In the foreseeable future, healthcare applications will continue to take a more active role in managing the medical responsibilities of attending physicians.\\
Types of Smartphone Applications for healthcare:\\
a) Patient medical health tracking apps.\\
b) Doctor appointment and clinical assistance apps.\\
c) Telehealth mobile apps.\\
d) Medical reference or database apps.\\
e) Monitoring apps for chronic conditions etc.\\

4. CONCLUSION:\\

The future of healthcare is shaping up in front of our very eyes with advances in digital healthcare technologies, such as artificial intelligence, VR/AR, 3D-printing, robotics or nanotechnology. We have to familiarize with the latest developments in order to be able to control technology and not the other way around. The future of healthcare lies in working hand-in-hand with technology and healthcare workers have to embrace emerging healthcare technologies in order to stay relevant in the coming years.\\
In medicine and healthcare, digital technology could help transform unsustainable healthcare systems into sustainable ones, equalize the relationship between medical professionals and patients, provide cheaper, faster and more effective solutions for diseases technologies could win the battle for us against cancer, AIDS or Ebola and could simply lead to healthier individuals living in healthier communities.



% Uncomment the following two lines if you want to have a bibliography
%\bibliographystyle{alpha}
%\bibliography{document}

\end{document}
